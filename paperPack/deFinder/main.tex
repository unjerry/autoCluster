\documentclass{article}
\usepackage{amsmath,amssymb,amsfonts}  % For math symbols and fonts
\usepackage{graphicx}                   % For including images
\usepackage{hyperref}                   % For hyperlinks
\usepackage{cite}                       % For citations

\usepackage{algorithm}
\usepackage{algorithmic}

% Title and author info
\title{Differential Informed Auto-Encoder}
\author{Zhang Jinrui\thanks{Liu Zhekai,Fu Xiangshuo} \\ \texttt{jerryzhang40@gmail.com}}

\date{20241021}  % Empty date; optional, you can also specify a date here

\begin{document}

\maketitle

\begin{abstract}
    In this article, an encoder was trained to obtain the inner
    structure of the original data by obtain a differential equations.
    A decoder was trained to resample from the domain of original data,
    to generate new data obey the differential structure of the original
    data using the Physics Informed Neural Network\cite[PINN]{raissi2017physics}.
\end{abstract}

\section{Introduction}
If the physics formula was obtained in the form of differential equations,
Physics Informed Neural network can be built to solve it numerically
in a global scale\cite[PINN]{raissi2017physics}.This process could be view
as a decoder in a way that it takes a sample point in the domain of
the partial differential equations, and solve it to get all the corresponding
output of each input point. If only a small and random amount of
training data was obtained, to resample from the domain we need
to obtain the differential relationship of the data. This process
could be viewed as an encoder that encodes the inner structure of
the original data. And the decoder decode it by solving the
differential equations.

\section{Methodology}
\subsection{first approach}
The first idea is simple. For a one variable function $u(t)$,
define a second order differential equations
in its general form $(\forall t)(F(\frac{d^2u}{{du}^2},\frac{du}{dt},u)=0)$.

The data of the function $u(t)$ are given in tuples denote as
$(T,U)_i\equiv(T_i,U_i)$. And It's natual to denote the differentials
as $U^{t}_{i}$ and $U^{tt}_{i}$. There are several method to compute
these two differentials, including just using the definition of
the derivative. In this article, local PCA are compute to obtain
these differentials. Local PCA means to find the K nearest neighbors
of a given point, which K is a hyperparameter, and perform PCA
on these points close to each other to get the principal direction.
The slope of this direction is the derivative $U^{t}$ in general.
Repeate these process on $(T,U^{t})$ to obtain $U^{tt}$

Create a FCN denote as $f$ to represent
$F(\frac{d^2u}{{du}^2},\frac{du}{dt},u)$
$F$ to be $0$ at every data points and to be $1$ all elsewhere is wanted.

To achieve these requirement we evaluate $f$ at all the data points,
and train the network evaluate these point to $0$. Then randomly sample
points from the $\mathbb{R}^{3}$ and train these points to be $1$.


\begin{algorithm}
    \caption{$f$ trainer}
    \begin{algorithmic}[1]
        \REQUIRE Input parameters $f,T_i, U_i, U^{t}_{i}, U^{tt}_{i}$
        % \ENSURE Output $z$
        \STATE Initialize $f$ randomly
        \REPEAT
        \STATE $F_i \leftarrow f(U_i,U^{t}_{i},U^{tt}_{i})$
        \STATE $RAND_i \leftarrow$ randomly sample in $\mathbb{R}^{3}$
        \STATE $R_i \leftarrow f(RAND_i)$
        \STATE $L \leftarrow meanSquareError(F_i,0)+0.1*meanSquareError(R_i,1)$
        \STATE backPropagation against $L$ to optimize $f$
        \UNTIL {$L$ meets requirement}
        % \FOR{each $i$ from $1$ to $n$}
        % \ENDFOR
        \RETURN $f$
    \end{algorithmic}
\end{algorithm}

Once The $f$ was obtained we can perform PINN as a decoder
to generate new data.


The experiment code for the Pics in Results can
be run by python by program in Github\cite[deSineTasks]{firstApproachGithubProject}
The requirement environment may be installed by using\cite[reqs]{Envreqs}

\section{Results}
\subsection{first approach}
Train the model on a pure $sin(x)$ and try to
get a result meet the initial condition with
$U^{t}_0=0.5$ which the $0.5*sin(x)$ would be
required output. Result shows in Figure \ref{fig:fig1}.

% \begin{figure}[h!]
%     \centering
%     \includegraphics[width=0.45\textwidth]{figde3/draw_2D__GeneredData.png}
%     \includegraphics[width=0.5\textwidth]{figde3/draw_2D__GeneredEqu.png}
%     \caption{generated data using the PINN to get $0.5*sin(x)$}
% \end{figure}
\begin{figure}[ht!]
    \centering
    \begin{minipage}{0.45\textwidth}
        \centering
        \includegraphics[width=0.9\textwidth]{figde3/draw_2D__GeneredData.png} % first figure itself
        \caption{$0.5*sin(x)$}
        \label{fig:fig1}
    \end{minipage}\hfill
    \begin{minipage}{0.45\textwidth}
        \centering
        \includegraphics[width=0.9\textwidth]{figde3/draw_2D__GeneredEqu.png} % second figure itself
        \caption{$f$ errors}
    \end{minipage}
\end{figure}

\section{Conclusion}
Summarize the key outcomes and potential future work.

% \begin{thebibliography}{99}
%     % Use \bibitem to reference your sources. Example:
%     \bibitem{example-ref} Author Name, \textit{Title of the Paper}, Journal, Year.
% \end{thebibliography}
\bibliographystyle{plain}  % or another style like unsrt, IEEEtran, etc.
\bibliography{references}  % references.bib is the file name

\end{document}
